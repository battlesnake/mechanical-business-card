\documentclass[12pt]{article}

\PassOptionsToPackage{svgnames}{xcolor}
\usepackage[margin=2cm]{geometry}
\usepackage{tikz}
\usepackage{amsmath}
\usepackage{amssymb}
\usetikzlibrary{trees}
\usetikzlibrary{decorations.markings}

\newenvironment{diagram} {
	\vspace{1ex}
	\begin{center}
	\begin{tikzpicture}
} {
	\end{tikzpicture}
	\end{center}
	\vspace{1ex}
}

\setlength\parskip{0.5ex}

\begin{document}

\section*{The system}

The system consists of the following components:

\begin{itemize}
	\setlength\itemsep{0ex}
	\item "Sun gear" in centre, with $N_s$ teeth and angular velocity $\omega_s$.
	\item "Planet gears" surrounding sun gear, with $N_p$ teeth and angular velocity $\omega_p$.
	\item "Annular gear" surrounding planet gears, with $N_a$ teeth and angular velocity $\omega_a$.
	\item "Planet carrier" which is fixed to each planet's axle with angular velocity $\omega_c$.
\end{itemize}

\begin{diagram}

\pgfkeys{/tikz/.cd,
  circle color/.initial=black,
  circle color/.get=\circlecolor,
  circle color/.store in=\circlecolor,
}
\tikzset{teeth/.style args={#1 and #2}{
   postaction=decorate,
   decoration={
    markings,
    mark=
    between positions 0 and 1 step #2
      with
      {
       \fill[radius=#1,\circlecolor] (0,0) circle;
      }
    }
  },
  teeth/.default={1pt and 3pt},
}
\tikzstyle{annular}=[draw=black, fill=none]
\tikzstyle{sun}=[draw=black, fill=yellow]
\tikzstyle{planet}=[draw=black, fill=SkyBlue]
\tikzstyle{gear}=[teeth]

\draw [annular,gear] (0,0) circle (5) node[anchor=south west] at (3.5,3.5) {$N_a$};
\draw [sun,gear] (0,0) circle (1) node {$N_s$};

\newcommand{\planetGeometry}[1]{
	\pgfmathsetmacro{\i}{#1}
	\pgfmathsetmacro{\r}{3}
	\pgfmathsetmacro{\c}{2.5}
	\pgfmathsetmacro{\C}{3.8}
	\pgfmathsetmacro{\l}{4.35}
	\pgfmathsetmacro{\t}{\i*120 + 15}
	\pgfmathsetmacro{\s}{\t-120}
	\pgfmathsetmacro{\px}{\r*cos(\t)}
	\pgfmathsetmacro{\py}{\r*sin(\t)}
	\pgfmathsetmacro{\lx}{\l*cos(\t)}
	\pgfmathsetmacro{\ly}{\l*sin(\t)}
	\pgfmathsetmacro{\u}{\c*cos(\s)}
	\pgfmathsetmacro{\v}{\c*sin(\s)}
	\pgfmathsetmacro{\U}{\C*cos(\s)}
	\pgfmathsetmacro{\V}{\C*sin(\s)}
	\pgfmathsetmacro{\x}{\c*cos(\t)}
	\pgfmathsetmacro{\y}{\c*sin(\t)}
	\pgfmathsetmacro{\X}{\C*cos(\t)}
	\pgfmathsetmacro{\Y}{\C*sin(\t)}
}
\foreach \i in {0,...,2} {
	\planetGeometry{\i}
	\draw [planet,gear] (\px,\py) circle (2) node {$N_p$} ;
	\draw [planet] (\lx,\ly) node {$N_p$} ;
}
\foreach \i in {0,...,2} {
	\planetGeometry{\i}
	\fill [blue] (\u,\v) -- (\x,\y) -- (\X,\Y) -- (\U,\V) -- cycle ;
	\draw (\u,\v) -- (\x,\y) ;
	\draw (\U,\V) -- (\X,\Y) ;
}

\end{diagram}

Input and output can be connected to any of the gears, or to the carrier.  If a planet gear is directly used for input or output, the system reduces to a simple 2-gear system.

In terms of reference frames, we have:

\tikzstyle{every node}=[thick,anchor=west, inner sep=3pt, rounded corners, draw=black]
\tikzstyle{noframe}=[thick,anchor=west, inner sep=3pt, draw=none]

\begin{diagram}[%
		grow via three points={one child at (0.5em,-2.0em) and two children at (0.5em,-2.0em) and (0.5em,-4.0em)},
		edge from parent path={(\tikzparentnode.south) |- (\tikzchildnode.west)}
	]
	\node {The system frame}
		child { node {The sun frame} }
		child { node {The carrier and planets frame}
			child { node {Planet 1 frame} }
			child { node [noframe] {\hspace{3.5ex}\vdots\hspace{3ex}\vdots\hspace{3ex}\vdots} }
			child { node {Planet $n$ frame} }
		}
		child { node at (0, -6em) {The annular frame} }
	;
\end{diagram}

\section*{Solving}

\subsection*{Fixed carrier}

In the case where the carrier is fixed, gear ratios and thus rotation rates are trivial to obtain:

\begin{align*}
	\omega_s : \omega_p &= N_p : -N_s\\
	\omega_p : \omega_a &= N_a : N_p\\
	\omega_s : \omega_a &= N_a : -N_s\\[0.5ex]
	\therefore\:\omega_s : \omega_p : \omega_a : \omega_c
		&= 1 : -\frac{N_s}{N_p} : -\frac{N_s}{N_a} : 0
\end{align*}

Note: all rotation rates are relative to orientation of parent frame.

\subsection*{Fixed annular gear}

To fix the annular gear instead of the carrier, we require that $\omega_c$ become free and that $\omega_a = 0$.  This is achieved by applying a rotation to the top-level frames to cancel out the rotation of the annular gear's frame.  We calculate the new rotation rates $\omega_x$ for all child frames of the system frame:

\def\wa{\frac{N_s}{N_a}}

\begin{align*}
	\omega_x &\rightarrow \omega^\prime_x - \left(-\frac{N_s}{N_a}\right)
		\textsf{\quad only for $\omega_x$ that are relative to the system frame}\\[1ex]
	\intertext{Note that $\omega_p$ is not changed since it is measured relative to the carrier frame, not the system frame.}
	\omega_s : \omega_p : \omega_a : \omega_c
		&= 1+\wa : -\frac{N_s}{N_p} : 0 : \wa\\[0.5ex]
		&= \frac{N_a+N_s}{N_a} : -\frac{N_s}{N_p} : 0 : \wa\\[1ex]
	\intertext{For convenience, normalize the sun gear:}
	\omega_s : \omega_p : \omega_a : \omega_c
		&= 1 : -\frac{N_sN_a}{N_p\left(N_a+N_s\right)} : 0 : \frac{N_s}{N_a+N_s}
\end{align*}

\end{document}